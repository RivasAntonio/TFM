\chapter{Conclusions}\label{ch:conclusiones}

To conclude, this master's thesis has brought us closer to the comprehension of criticality emerging from Hawkes processes. We have fulfilled the objectives set at the beginning of the project, 
starting with a brief overview of criticality in complex systems, with examples in different fields, such as physics or living systems. After a short introduction to point processes,
we understood Hawkes processes themselves, how they can be used to model self-exciting events, and how they can be simulated. We also have seen how criticality arises from 
these processes as long as suitable parameters and time binning are chosen. 
We have developed computational tools to reproduce the results of the paper by \cite{notarmuzi2021percolation} satisfactorily. Moreover, these tools have allowed us to extend the analysis
to other configurations, such as the supercritical regime, and realising that the process ceases to be critical. 

Furthermore, we also have studied other configurations other than the one presented in the paper, obtaining valuable insights into the system dynamics, specifically, for the case of 
$n=2$ there is not intrinsic criticality in the system, the one observed is due to the time binning method. 
Finally, we have extended the model 
to account for both excitatory and inhibitory dynamics, and we have repeated the analysis based on the time-binning method to seek for any fingerprints of criticality 
and to approach a more realistic model of the brain. Further research could be the analysis of the parameter space of two coupled
Hawkes processes, the examination of the statistics of excitation and inhibition separately, the study of more processes or the influence of a network structure in the dynamics of the system.  
