\chapter{Conclusions}\label{ch:conclusiones}

To conclude, this master's thesis has brought us closer to the comprehension of criticality emerging from Hawkes processes. We have fulfilled the objectives set at the beginning of the project, 
starting with a brief overview of criticality in complex systems, with examples in different fields, such as physics or living systems. After a short introduction to point processes,
we understood Hawkes processes themselves, how they can be used to model self-exciting events, and how they can be simulated. We also have seen how criticality arises from 
these processes as long as suitable parameters and time binning are chosen. We have developed computational tools to reproduce the results of the paper by \cite{notarmuzi2021percolation} 
satisfactorily. Furthermore, we also have studied other configurations other than the one presented in the paper, obtaining valuable insights into the system dynamics. Finally, we have 
begun to explore the coupling of these processes in order to approach a more realistic model of the brain. Further research could be the analysis of the parameter space of two coupled
Hawkes processes, the examination of the statistics of excitation and inhibition separately, the study of more processes or the influence of a network structure in the dynamics of the system.  
