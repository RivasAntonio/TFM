\chapter{Methodology}\label{ch:metodologia}

In the following sections, the methodology for data generation, managment and analysis will be presented. To adress these issues, we will use Python \cite{McKinney,VanderPlas} due to 
its versatility and the wide range of libraries available. The two used will be NumPy \cite{numpy} and Matplotlib \cite{matplotlib} for the visualization.
 
\section{Time series factory}\label{sec:time_series_factory}

The first step is the generation of time series, there are two ways to do this: the slow one and the fast one. The first one is discretizing the time and calculating the rate at each time step
according with Eq \ref{eq: Hawkes rate}, then accept or reject the event if $p\lambda \cdot dt$ for a random number $p\in \mathcal{U}[0,1]$ . 
This method works for small time series, but for large ones is not efficient because the summation of the kernel function has to be done at each time step. The pseudo-code for this method is
presented in Algorithm \ref{alg: slow method}.

\begin{algorithm}
    \caption{Slow method to generate Hawkes processes.}\label{alg: slow method}
    \begin{algorithmic}
        \Require $t_max$, $n_{intervals}$, $\lambda(t_0)=\mu$, $p$
        \State $dt \gets \frac{t_{max}}{n_{intervals}}$
        \For {$i=0$ to $n_{intervals}$}
            \State $\lambda(t_k) \gets \mu + n\sum_{t_i<t_k}\phi(t_k-t_i)$ \Comment{$t_i=i\cdot dt$}
            \If {$\lambda(t_k)\cdot dt > p$}
                \State $t_{event} \gets t_k$
            \EndIf
        \EndFor
        \end{algorithmic}
\end{algorithm}

The fast method takes advantage of Monte Carlo methods \cite{barbu2020monte} to generate the time series. The idea of this procedure consists in computing the inter-event time instead 
of the time of the event. To get to the algorithm, we start from the following expression:

\begin{equation}
    PDF(\text{inter-event time}=\Delta t) = \lambda(t+\Delta t) e^{-\int_t^{t+\Delta t}\lambda(t^\prime)dt^\prime} 
    \label{eq: inter-event time PDF}
\end{equation}

To demonstrate this, we have to take a look at the Figure \ref{f: Figura calculo probabilidad acumulada} and recall that $\lambda$ is a probability per unit of time.

\begin{figure}[H]
    \centering
    \includegraphics[width=0.7\textwidth]{Figura cálculo probabilidad acumulada.png}
    \caption{Diagram to calculate the cumulative probability of the inter-event time.}
    \label{f: Figura calculo probabilidad acumulada}
\end{figure}

The probability per unit of time of having an event in the interval $[t+\Delta t,t+\Delta t+dt]$ is the probability of no events in the interval $[t,t+\Delta t]$ times the probability 
of happening in the interval $[t+\Delta t,t+\Delta t+ dt]$. Putting words into mathematics, we have that the probability of not having an event in the interval $[t,t+\Delta t]$ is:
\begin{equation}
    \begin{split}
        P(\text{event}\in [t+\Delta t+dt])=& \left(1-\lambda(0)\cdot dt \right)\left(1-\lambda(dt)\cdot dt \right)\left(1-\lambda(2dt)\cdot dt \right)\ldots\\
        =& \prod_{k=0}\underbrace{\left(1 -\lambda(k dt)\cdot dt \right)}_{e^{\ln \left(1-\lambda(k dt)dt \right)}}=e^{\sum_{k=0}\ln \left(1-\lambda(k dt) dt \right)}= \ldots\quad \text{Using}  \ln(1-\varepsilon)\approx -\varepsilon\\
        =& e^{-\sum_{k=0}\lambda(kdt)dt}\underbrace{=}_{dt\to 0}e^{-\int_{t}^{t+\Delta t}\lambda(t^\prime)dt^\prime}.
    \end{split}
    \label{eq: probability of no events}
\end{equation} 
Knowing that the probability of having an event in the interval $[t+\Delta t,t+\Delta t+dt]$ is $\lambda(t+\Delta t)dt$, we have:

\begin{equation}
P(\text{event}\in [t+\Delta t,t+\Delta t+dt])\cancel{dt}= \lambda(t+\Delta t)dt\cdot e^{-\int_{t}^{t+\Delta t}\lambda(t^\prime)dt^\prime}PDF(\text{inter-event time}=\Delta t)\cancel{dt}.
\label{eq: probability of having an event}
\end{equation}

Having that we can calculate the inter-event time following the next steps. 

$$ \text{PDF}\left( \text{inter-event time } = \Delta,t \right) = \lambda(t+\Delta t)\underbrace{e^{\int_{t}^{t+\Delta t}\lambda(t^{\prime}dt^\prime)}}_{\text{No events during }(t,t+\Delta t)}$$

In order to generate $\Delta t$, we will use the inverse transform method \cite{Toral}, therefore we have to calculate the cumulative probability of the inter-event time:
\begin{equation}
    \begin{split}
        \text{accum}(\Delta t)=&\int_{0}^{\Delta t}\text{PDF}\left( \Delta t^\prime \right)d\Delta t^\prime = u\in\mathcal{U}  [0,1]\\
        &\int_{0}^{\Delta t} \underbrace{\lambda(t+\Delta t^\prime)e^{\int_{t}^{t+\Delta t^\prime}\lambda(t^{\prime})dt^{\prime}}}_{-\dfrac{d}{d\Delta t^{\prime}}\left[ e^{-\int_{t}^{t+\Delta t}\lambda t^\prime dt^\prime} \right]}d\Delta t^\prime=u \qquad \text{Using Barrow rule}\\
        &-e^{-\int_{t}^{t+\Delta t}\lambda(t^\prime)dt^\prime}\Big|_{0}^{\Delta t}=1-e^{-\int_{t}^{t+\Delta t}\lambda(t^\prime)dt^\prime}=u\qquad \text{Taking logarithms}\\
        &\int_{t}^{t+\Delta t}\lambda(t^\prime)dt^\prime=-\ln(1-u) = \ln (\bar{u})\\
    \end{split}
    \label{eq: cumulative probability}
\end{equation}

To compute the inter-event time, we have to generate $\bar{u}\sim$ and solve the equation. Having in mind this relation and using Eq \ref{eq: Hawkes rate at event time} we have:
EN EL LA ECUACIÓN REFERENCIADA Y LA SEGUNDA EXPONENCIAL SE PUEDE PONER $\lambda(t_k^-)$ en lugar de $\lambda(t_k)$? .
\begin{equation}
    \begin{split}
        u=&1-e^{-\mu(t-t_k)}e^{-(\lambda(t_k)+\alpha-\mu)\cdot\overbrace{\int_{t_k}^{t}e^{-\beta(t^\prime)-t_k}dt^\prime}^{-\frac{1}{\beta}\left[ e^{-\beta}(t-t_k)-1 \right]}}\\
        u=&1-\underbrace{e^{-\mu(t-t_k)}}_{P(t_{k+1}^{(1)}>t)}\underbrace{e^{-\left[ (\lambda(t_k)+\alpha-\mu)\beta^{-1}\left( 1-e^{-\beta(t-t_k)} \right) \right]}}_{{P(t_{k+1}^{(2)}>t)}}
    \end{split}
    \label{eq: u = 1 - P(t_{k+1} > t)P(t_{k+1} > t), previo al composition method}
\end{equation}

Then we apply the composition method \cite{dassios2013exact}. If we take $t_{k+1}=\min(t_{k+1}^(1),t_{k+1}^(2))$; then $t_{k+1}\sim P(t_{k+1}>t)$, hence:
\begin{equation}
    \begin{split}
        \text{Prob}(t_{k+1}=\min\left(  t_{k+1}^{(1)},t_{k+1}^{(2)} \right)\leq t)=&1-\text{Prob}\left( \min \left( t_{k+1}^{(1)},t_{k+1}^{(2)}\right)>t \right)\\
        =&1-\text{Prob}\left( t_{k+1}^{(1)}>t \right)\cdot\text{Prob}\left( t_{k+1}^{(2)}>t \right)\\
    \end{split}
    \label{eq: composition method}
\end{equation}
where we have used that the probability that the smaller is greater than $t$ is that each separately is greater than $t$ because both have to be greater than $t$. As we can 
see the expressions in Eqs \ref{eq: u = 1 - P(t_{k+1} > t)P(t_{k+1} > t), previo al composition method} and \ref{eq: composition method} are the same, so we can use the composition 
method to generate the inter-event time. Then, the algorithm to generate the inter-event time is:
\begin{enumerate}
    \item Generate $t_{k+1}^{(1)}\sim P \left( t_{k+1}^{(1)}>t \right) = e^{-\mu\left( t-t_k \right)}$ using 
    $$P\left( t_{k+1}^{(1)}\leq t \right) = 1- \underbrace{e^{-\mu(t-t_k)}}_{\bar{u_1}\in\mathcal{U}[0,1]\Rightarrow =u_1-\bar{u_1}}=u_1 \in \mathcal{U}[0,1]$$
    This is done by generating $u_1\in\mathcal{U}[0,1]$ and solving the equation.
    \begin{equation}
        \begin{split}
        &u_1=1-e^{-\mu\left( t_{k+1}^{(1)}-t_k \right)}\\
        &\ln(u_1)=-\mu\left( t_{k+1}^{(1)}-t_k  \right)\Rightarrow t_{k+1}^{(1)}=t_k-\dfrac{\ln(u_1)}{\mu}            
        \end{split}
        \label{eq: inter-event time 1}
    \end{equation}
    \item Generate $t_{k+1}^{(2)}\sim P\left( t_{k+1}^{(2)}>t \right)=e^{-\left( \left( \lambda(t_k)+\alpha-\mu \right)\beta^{-1}\left( 1-e^{-\beta\left( t_{k+1}^{(2)-t_k}\right)}\right)\right)}$
    in a similar way as before:
    \begin{equation}
        \begin{split}
            u_2=&1-e^{-\left( \left( \lambda(t_k)+\alpha-\mu \right)\beta^{-1}\left( 1-e^{-\beta\left( t_{k+1}^{(2)-t_k}\right)}\right)\right)}\\
            -\ln(u_2)=&\left( \left( \lambda(t_k)+\alpha-\mu \right)\beta^{-1}\left( 1-e^{-\beta\left( t_{k+1}^{(2)-t_k}\right)}\right)\right)\\ 
            & 1+\dfrac{\beta\ln u_2}{\lambda(t_k)+\alpha-\mu}=e^{-\beta\left( t_{k+1}^{(2)-t_k}\right)}\\
            t_{k+1}^{(2)}=&t_k-\beta^{-1}\ln\underbrace{\left( 1+\dfrac{\beta\ln u_2}{\lambda(t_k)+\alpha-\mu} \right)}_{\text{This number must be positive}}   
        \end{split}
        \label{eq: inter-event time 2}
    \end{equation}   
    \item Choose $t_{k+1}=\min\left( t_{k+1}^{(1)},t_{k+1}^{(2)} \right)$
    \item Calculate the rate at $t_{k+1}$ using Eq \ref{eq: Hawkes rate exponential becomes Markovian} and go back to step 1.
\end{enumerate}

With this method, we can generate time series efficiently. The pseudo-code for this method is presented in Algorithm \ref{alg: fast method}.
\begin{algorithm}
    \caption{Algorithm to generate $K$ Hawkes events.}\label{alg: fast method}
    \begin{algorithmic}
        \Require $\alpha$, $\beta$, $\lambda(t_0)=\mu$, $K$
        \For {$k=0$ to $K$}
            \State $u_1,u_2 \gets \mathcal{U}[0,1]$
            \State $t_{k+1}^{(1)} \dfrac{\ln(u_1)}{\mu}$
            \State $t_{k+1}^{(2)} \beta^{-1}\ln\left( 1+\dfrac{\beta\ln u_2}{\lambda(t_k)+\alpha-\mu} \right)$
            \State $t_{k+1} \gets \min\left( t_{k+1}^{(1)},t_{k+1}^{(2)} \right)$
            \State $\lambda(t_{k+1}) \gets \mu + e^{-\beta(t_{k+1}-t_k)}\left( \lambda(t_k)-\mu+n \right)$
        \EndFor
    \end{algorithmic}
\end{algorithm}


\section{When physics and cooking merge}\label{sec:physics_cooking}