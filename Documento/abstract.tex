\chapter*{Abstract}
\addcontentsline{toc}{chapter}{Abstract}
In this master thesis we have explored the relationship between Hawkes processes and criticality in living systems. These processes are an exceptional tool to
model self-exciting events, which are relevant in several contexts, such as for the brain, where neurons excite or inhibit each other eliciting cascades of events, in earthquakes, 
where a seismic event can trigger another one, or in social media, where a post can provoke reactions in the audience.
We begin by reviewing stochastic processes, specifically point processes, different examples of them, and where they can be found. Next, in order to understand the dynamics of self-exciting 
phenomena, we focus on Hawkes processes.

In this thesis, we developed computational methods to generate these processes and analyse their behaviour. We have studied the critical behaviour of a single Hawkes process,
the supercritical behaviour of a self-exciting process, and the dynamics of an excitatory and inhibitory coupled Hawkes process. For each of these cases, several sets of parameters have been
explored, both reproducing known results and obtaining new valuable insights.
These findings give us a better understanding of the criticality underlying these processes and their applications in numerous living systems. Moreover, in the last section of the project
we propose future research to shed light on the influence of coupling in the dynamics of the system.

\chapter*{Resumen}
\addcontentsline{toc}{chapter}{Resumen}

En este trabajo fin de máster hemos estudiado la relación entre los procesos de Hawkes y la criticidad en sistemas vivos. Estos procesos son excepcionales para la modelización de eventos 
autoexcitados, como haber en el cerebro, donde las neuronas pueden excitarse o inhibirse entre sí, en terremotos, donde un seísmo puede provocar otro, o en redes sociales, donde una 
publicación puede desencadenar reacciones en su audiencia. Este proyecto comienza dando una introducción a los procesos estocásticos, concretamente a los procesos puntuales, presentándose 
diferentes ejemplos junto a sus aplicaciones. Debido al objetivo principal de la investigación, nos centramos en los procesos de Hawkes, para comprender la dinámica de los fenómenos autoexcitados.

Se han desarrollado métodos computacionales para generar estos procesos y analizar su comportamiento. Se ha estudiado el comportamiento crítico de un único proceso de Hawkes, el comportamiento
supercrítico y la dinámica de un proceso de Hawkes excitador acoplado con uno inhibidor. Para cada una de estas configuraciones se han investigado varios conjuntos de parámetros, reproduciendo 
los resultados ya conocidos y obteniendo nuevos conocimientos para las nuevas configuraciones. Estos descubrimientos nos proporcionan una mejor comprensión de la criticidad que aparece en 
estos procesos y sus aplicaciones en diferentes sistemas vivos. Además, la última sección abre la posibilidad de futuras investigaciones para comprender la influencia del acople
de procesos de Hawkes en la dinámica del sistema.