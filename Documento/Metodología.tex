\chapter{Methodology}\label{ch:metodologia}

In the following sections, the methodology for data generation, management and analysis will be presented. To address these issues, we will use Python \cite{McKinney,VanderPlas} due to 
its versatility and the wide range of libraries available. The two used will be NumPy \cite{numpy} and Matplotlib \cite{matplotlib} for the visualization.
 
\section{Time series factory}\label{sec:time_series_factory}

The first step is the generation of time series. There are two ways to do this: the slow one and the fast one. The first one is discretizing the time and calculating the rate at each time step
according to Eq. \ref{eq: Hawkes rate}, then accept or reject the event if $p<\lambda \cdot dt$ for a random number $p\in \mathcal{U}[0,1]$ . 
This method works for small time series, but for large ones is not efficient because the summation of the kernel function has to be done at each time step. The pseudo-code for this method is
presented in Algorithm \ref{alg: slow method}.

\begin{algorithm}
    \caption{Slow method to generate Hawkes processes.}\label{alg: slow method}
    \begin{algorithmic}
        \Require $t_max$, $n_{intervals}$, $\lambda(t_0)=\mu$, $p$
        \State $dt \gets \frac{t_{max}}{n_{intervals}}$
        \For {$i=0$ to $n_{intervals}$}
            \State $\lambda(t_k) \gets \mu + n\sum_{t_i<t_k}\phi(t_k-t_i)$ \Comment{$t_i=i\cdot dt$}
            \If {$\lambda(t_k)\cdot dt > p$}
                \State $t_{event} \gets t_k$
            \EndIf
        \EndFor
        \end{algorithmic}
\end{algorithm}

The fast method takes advantage of Monte Carlo methods \cite{barbu2020monte} to generate the time series. The idea behind this procedure is to compute the inter-event time instead 
of the time of the event. To get to the algorithm, we start from the following expression:

\begin{equation}
    PDF(\text{inter-event time}=\Delta t) = \lambda(t+\Delta t) e^{-\int_t^{t+\Delta t}\lambda(t^\prime)dt^\prime} 
    \label{eq: inter-event time PDF}
\end{equation}

To demonstrate this, we have to take a look at Figure \ref{f: Figura calculo probabilidad acumulada} and recall that $\lambda$ is a probability per unit of time.

\begin{figure}[H]
    \centering
    \includegraphics[width=0.7\textwidth]{Figura cálculo probabilidad acumulada.png}
    \caption{Diagram to calculate the cumulative probability of the inter-event time.}
    \label{f: Figura calculo probabilidad acumulada}
\end{figure}
The probability that the inter-event interval is $\Delta t$ is equal to the probability that no events occur in the interval $[t,t+\Delta t]$  times the probability that an event occur in the 
interval $[t+\Delta t,t+\Delta t+ dt]$. Putting words into mathematics, we have that the probability of not having an event in the interval $[t,t+\Delta t]$ is:
\begin{equation}
    \begin{split}
        P(\text{No event}\in [t+\Delta t+dt])=& \left(1-\lambda(0)\cdot dt \right)\left(1-\lambda(dt)\cdot dt \right)\left(1-\lambda(2dt)\cdot dt \right)\ldots\\
        =& \prod_{k=0}\underbrace{\left(1 -\lambda(k dt)\cdot dt \right)}_{e^{\ln \left(1-\lambda(k dt)dt \right)}}=e^{\sum_{k=0}\ln \left(1-\lambda(k dt) dt \right)}= \ldots\quad \text{Using}  \ln(1-\varepsilon)\approx -\varepsilon\\
        =& e^{-\sum_{k=0}\lambda(kdt)dt}\underbrace{=}_{dt\to 0}e^{-\int_{t}^{t+\Delta t}\lambda(t^\prime)dt^\prime}.
    \end{split}
    \label{eq: probability of no events}
\end{equation} 
Knowing Eq. \ref{eq: inter-event time PDF} we can calculate the inter-event time following the next steps. 
In order to generate $\Delta t$, we will use the inverse transform method \cite{Toral}, therefore, we have to calculate the cumulative probability of the inter-event time:
\begin{equation}
    \begin{split}
        \text{accum}(\Delta t)=&\int_{0}^{\Delta t}\text{PDF}\left( \Delta t^\prime \right)d\Delta t^\prime = u\in\mathcal{U}  [0,1]\\
        &\int_{0}^{\Delta t} \underbrace{\lambda(t+\Delta t^\prime)e^{-\int_{t}^{t+\Delta t^\prime}\lambda(t^{\prime})dt^{\prime}}}_{-\dfrac{d}{d\Delta t^{\prime}}\left[ e^{-\int_{t}^{t+\Delta t^\prime}\lambda (t^\prime) dt^\prime} \right]}d\Delta t^\prime=u \qquad  \text{Using Barrow rule}\\
        &-e^{-\int_{t}^{t+\Delta t^\prime}\lambda(t^\prime)dt^\prime}\Big|_{0}^{\Delta t^\prime}=1-e^{-\int_{t}^{t+\Delta t}\lambda(t^\prime)dt^\prime}=u\qquad  \text{Taking logarithms}\\
        &\int_{t}^{t+\Delta t}\lambda(t^\prime)dt^\prime=-\ln(1-u) = \ln (\bar{u})\\
    \end{split}
    \label{eq: cumulative probability}
\end{equation}

To compute the inter-event time, we have to generate $\bar{u}\sim\mathcal{U}[0,1]$ and solve the equation. Having in mind this relation and using Eq. \ref{eq: Hawkes rate at event time} we have:

\begin{equation}
    \begin{split}
        u=&1-e^{-\mu(t-t_k)}e^{-(\lambda(t_k)+\alpha-\mu)\cdot\overbrace{\int_{t_k}^{t}e^{-\beta(t^\prime-t_k)}dt^\prime}^{-\frac{1}{\beta}\left[ e^{-\beta}(t-t_k)-1 \right]}}\\
        u=&1-\underbrace{e^{-\mu(t-t_k)}}_{P(t_{k+1}^{(1)}>t)}\underbrace{e^{-\left[ (\lambda(t_k)+\alpha-\mu)\beta^{-1}\left( 1-e^{-\beta(t-t_k)} \right) \right]}}_{{P(t_{k+1}^{(2)}>t)}}
    \end{split}
    \label{eq: u = 1 - P(t_{k+1} > t)P(t_{k+1} > t), previo al composition method}
\end{equation}

Where we have interpreted these two factors as probabilites (because they are smaller than one) of the inter-event time being greater than $t$. This is the key to generate the inter-event time
because if we do not decompose them, we can not invert the function.
Then we apply the composition method \cite{dassios2013exact}. If we take $t_{k+1}=\min\left( t_{k+1}^{(1)},t_{k+1}^{(2)} \right)$; then $t_{k+1}\sim P(t_{k+1}>t)$, hence:
\begin{equation}
    \begin{split}
        \text{Prob}(t_{k+1}=\min\left(  t_{k+1}^{(1)},t_{k+1}^{(2)} \right)\leq t)=&1-\text{Prob}\left( \min \left( t_{k+1}^{(1)},t_{k+1}^{(2)}\right)>t \right)\\
        =&1-\text{Prob}\left( t_{k+1}^{(1)}>t \right)\cdot\text{Prob}\left( t_{k+1}^{(2)}>t \right)\\
    \end{split}
    \label{eq: composition method}
\end{equation}
where we have used that the probability that the smaller is greater than $t$ is that each separately is greater than $t$. As we can 
see, the expressions in Eqs. \ref{eq: u = 1 - P(t_{k+1} > t)P(t_{k+1} > t), previo al composition method} and \ref{eq: composition method} are the same, so we can use the composition 
method to generate the inter-event time. Then, the algorithm to generate the inter-event time is:
\begin{enumerate}
    \item Generate $t_{k+1}^{(1)}\sim P \left( t_{k+1}^{(1)}>t \right) = e^{-\mu\left( t-t_k \right)}$ using 
    $$P\left( t_{k+1}^{(1)}\leq t \right) = \underbrace{1- \underbrace{e^{-\mu(t-t_k)}}_{\bar{u_1}\in\mathcal{U}[0,1]}}_{1-\bar{u_1}=u_1}=u_1 \in \mathcal{U}[0,1]$$
    This is done by generating $u_1\in\mathcal{U}[0,1]$ and solving the equation.
    \begin{equation}
        \begin{split}
        &u_1=1-e^{-\mu\left( t_{k+1}^{(1)}-t_k \right)}\\
        &\ln(u_1)=-\mu\left( t_{k+1}^{(1)}-t_k  \right)\Rightarrow t_{k+1}^{(1)}=t_k-\dfrac{\ln(u_1)}{\mu}            
        \end{split}
        \label{eq: inter-event time 1}
    \end{equation}
    \item Generate $t_{k+1}^{(2)}\sim P\left( t_{k+1}^{(2)}>t \right)=e^{-\left( \left( \lambda(t_k)+\alpha-\mu \right)\beta^{-1}\left( 1-e^{-\beta\left( t_{k+1}^{(2)}-t_k\right)}\right)\right)}$
    in a similar way as before:
    \begin{equation}
        \begin{split}
            &u_2=1-e^{-\left( \left( \lambda(t_k)+\alpha-\mu \right)\beta^{-1}\left( 1-e^{-\beta\left( t_{k+1}^{(2)}-t_k\right)}\right)\right)}\\
            -&\ln(u_2)=\left( \left( \lambda(t_k)+\alpha-\mu \right)\beta^{-1}\left( 1-e^{-\beta\left( t_{k+1}^{(2)}-t_k\right)}\right)\right)\\ 
            & 1+\dfrac{\beta\ln u_2}{\lambda(t_k)+\alpha-\mu}=e^{-\beta\left( t_{k+1}^{(2)-t_k}\right)}\\
            &t_{k+1}^{(2)}=t_k-\beta^{-1}\ln\underbrace{\left( 1+\dfrac{\beta\ln u_2}{\lambda(t_k)+\alpha-\mu} \right)}_{\text{This number must be positive}}   
        \end{split}
        \label{eq: inter-event time 2}
    \end{equation}   
    \item Choose $t_{k+1}=\min\left( t_{k+1}^{(1)},t_{k+1}^{(2)} \right)$
    \item Calculate the rate at $t_{k+1}$ using Eq. \ref{eq: Hawkes rate exponential becomes Markovian} and go back to step 1.
\end{enumerate}

With this method, we can generate time series efficiently. The pseudo-code for this method is presented in Algorithm \ref{alg: fast method}.
\begin{algorithm}
    \caption{Algorithm to generate $K$ Hawkes events.}\label{alg: fast method}
    \begin{algorithmic}
        \Require $\alpha$, $\beta$, $\lambda(t_0)=\mu$, $K$
        \For {$k=0$ to $K$}
            \State $u_1,u_2 \gets \mathcal{U}[0,1]$
            \State $t_{k+1}^{(1)}\gets \dfrac{\ln(u_1)}{\mu}$
            \State $t_{k+1}^{(2)} \gets \beta^{-1}\ln\left( 1+\dfrac{\beta\ln u_2}{\lambda(t_k)+\alpha-\mu} \right)$
            \State $t_{k+1} \gets \min\left( t_{k+1}^{(1)},t_{k+1}^{(2)} \right)$
            \State $\lambda(t_{k+1}) \gets \mu + e^{-\beta(t_{k+1}-t_k)}\left( \lambda(t_k)-\mu+n \right)$
        \EndFor
    \end{algorithmic}
\end{algorithm}

To conclude the time series generation section, we can generalize the algorithm in order to generate $M$ coupled Hawkes processes \cite{dassios2013exact,laub2021elements}. 
The essence of the algorithm is the 
same as the one presented. First, we generate the inter-event time for the excitatory population and the inhibitory population, after that, we choose the minimum of both and update the
rates of both populations according to the event that has just occurred. Mathematically, it is expressed as follows:

\begin{enumerate}
    \item Generate $\Delta_{k+1} = \min\left\{ \Delta_{k+1}^{(1)},\Delta_{k+1}^{(2)} \right\}$ with $\Delta_{k+1}^{(j)}=t_{k+1}^{(j)}-t_k^{(j)}$ generated 
    as in Eqs. \ref{eq: inter-event time 1} and \ref{eq: inter-event time 2}.
    \begin{equation}
        \Delta_{k+1}^{(j)} = \min \left\{ -\dfrac{\ln(u_1^{(j)})}{\mu_j},-\beta_j^{-1}\ln\left( \underbrace{1+\dfrac{\beta_j\ln u_2^{(j)}}{\lambda_j\left( t_k^{(j)} \right)+\alpha_j-\mu_j}}_{g_j} \right) \right\}
        \label{eq: inter-event time coupled}
    \end{equation}
    Note that $g_j$ must be positive, otherwise, take the other term.
    \item Once we have the process $(l)$, we update the time for the following event as $t_{k+1}=t_k+\Delta_{k+1}^{(l)}$.
    \item Update the rates for the excitatory and inhibitory populations as follows:
    \begin{equation}
        \lambda_j(t_{k+1}) = \mu_j + e^{-\beta_j(t_{k+1}-t_k)}\left( \lambda_j(t_k)-\mu_j+\alpha_{l\to j} \right) \qquad \text{with } j=1,2
    \end{equation}
\end{enumerate}

The pseudo-code for this method is presented in Algorithm \ref{alg: fast method coupled}.

\begin{algorithm}[H]
    \caption{Algorithm to generate $K$ Hawkes events for two coupled processes.}\label{alg: fast method coupled}
    \begin{algorithmic}
        \Require $\alpha_{11}$, $\alpha_{12}$, $\beta_1$, $\mu_1$, $\alpha_{22}$, $\alpha_{21}$, $\beta_2$, $\mu_2$, $K$
        \For {$k=0$ to $K$}
            \State $u_1^{(1)},u_2^{(1)},u_1^{(2)},u_2^{(2)} \gets \mathcal{U}[0,1]$
            \State $\Delta_{k+1}^{(1)}\gets \min\left( -\dfrac{\ln(u_1^{(1)})}{\mu_1},-\beta_1^{-1}\ln\left( 1+\dfrac{\beta_1\ln u_2^{(1)}}{\lambda_1(t_k)+\alpha_{11}-\mu_1} \right) \right)$
            \State $\Delta_{k+1}^{(2)}\gets \min\left( -\dfrac{\ln(u_1^{(2)})}{\mu_2},-\beta_2^{-1}\ln\left( 1+\dfrac{\beta_2\ln u_2^{(2)}}{\lambda_2(t_k)+\alpha_{22}-\mu_2} \right) \right)$
            \State $l\gets \arg\min\left( \Delta_{k+1}^{(1)},\Delta_{k+1}^{(2)} \right)$
            \State $t_{k+1}\gets t_k+\Delta_{k+1}^{(l)}$
            \State $\lambda_1(t_{k+1}) \gets \mu_1 + e^{-\beta_1(t_{k+1}-t_k)}\left( \lambda_1(t_k)-\mu_1+\alpha_{l\to 1} \right)$
            \State $\lambda_2(t_{k+1}) \gets \mu_2 + e^{-\beta_2(t_{k+1}-t_k)}\left( \lambda_2(t_k)-\mu_2+\alpha_{l\to 2} \right)$
        \EndFor
    \end{algorithmic}
\end{algorithm}


\section{The importance of time binning}\label{sec:physics_cooking}

Now we have a method to generate the main ingredient, time series. In order to cook (analyze) them, our tools will be Python libraries and a control parameter, which in our case will be 
a resolution parameter $\Delta >0$ that will allow us to identify clusters of activity. Assuming that we have a time series with $K$ events that happen in times $\left\{ t_1,\ldots, t_K \right\}$.
Each event starts a cluster, and the following event will be part of this cluster if the time between both events is less than $\Delta$ and so on for the rest. We define the size of the cluster 
as the number of events in the interval $[t_{first},t_{last}]$ and its duration as $t_{last}-t_{first}$. The extreme cases are when $\Delta$ is smaller than the minimum inter-event time, 
where each event is a cluster of size 1 and duration 0. The other extreme is when $\Delta$ is greater than the largest inter-event time, where all the events are in the same cluster of size $K$
and duration $t_K-t_1$. Between these two extremes, we will have different regimes of the process. Our recipe will be the phase diagram, specifically the percolation diagram, where we will
plot the percolation strength $P_{\infty}$ as a function of the resolution parameter $\Delta$. The percolation strength is defined as the fraction of events that are in the largest cluster over
the total number of events. Three different sets of parameters will be used to generate the time series in order to compare them. The parameters $\alpha$ and $\beta$ will be fixed to unless 
otherwise stated, the other parameters are shown in Table \ref{tab: parameters}. Once we got our recipe (the percolation diagram), we should be able to identify the critical points and the
different regimes of the process. By carefully choosing  $\Delta$ from the percolation diagram, we will be able to identify the different regimes with their respective power law exponents. 


\begin{table}[H]
    \centering
    \caption{Configuration of the parameters for the simulations of the article \cite{notarmuzi2021percolation}.}
    \label{tab: parameters}
    \begin{tabular}{@{}lcc@{}}
    \toprule
    Configuration & \multicolumn{1}{c}{$\mu$} & \multicolumn{1}{c}{$n$} \\ \midrule
    First & 1 & 0 \\
    Second & $10^{-4}$ & 1 \\
    Third & $10^2$ & 1 \\ \bottomrule
    \end{tabular}
\end{table}

The percolation diagram will be generated by generating 1000 time series for each configuration and calculating the percolation strength for each one becausethey are stochastic
processes, as we can observe in Figure \ref{f: Hawkes not stationary}.

\begin{figure}[H]
    \centering
    \includegraphics[width=0.9\textwidth]{Signals.png}
    \caption{Five a temporal series of $K=10^5$ events of Hawkes processes with $\mu=10^{-4}$ on the left side and $\mu = 10^2$ on the right one.}
    \label{f: Hawkes not stationary}
\end{figure}

Beginning with the first configuration, we have a homogeneous Poisson process, and the inter-event probability of having an inter-event
time $x_i$ is given by $P(x_i)=\mu e^{-\mu x_i}$. Consequently, two consecutive events will be part of a cluster, fixing the resolution parameter to $\Delta$ with a probability of
\begin{equation}
    P(x_i\leq \Delta)=1-e^{-\mu\Delta}\quad\quad \forall i.
    \label{eq: Poisson prob of cluster of size 2}
\end{equation}

This represents the probability in a homogeneous 1D percolation model \cite{stauffer2018introduction}, where we can identify a non percolant phase and a percolant phase separated by the
critical point $\Delta^*$. We can calculate this parameter if we know the maximum inter-event time of the time series. Let us assume that our time series has $K$ events. Therefore, it will 
percolate if the condition we have just stablished is satisfied. We can calculate this threshold as the average of the maximum inter-event time in $K$ samples from the 
inter-event time distribution solving the following equation:
\begin{equation}
    \begin{split}
        &K\int_{\Delta^*}^{\infty}P(x) dx=1\\
        &K\int_{\Delta^*}^{\infty}\mu e^{-\mu x}dx=1\\
        &-K\left[ e^{-\mu x} \right]_{\Delta^*}^{\infty}=K\left[ e^{-\mu\Delta^*} - \cancelto{0}{e^{-\mu\infty}} \right]=1\\
        &K e^{-\mu\Delta^*}=1\\
        &\Delta^*(K)=-\dfrac{\ln\left(K \right)}{\mu}
    \end{split}
    \label{eq: critical point}
\end{equation}

For the other two configurations, on both we have a self-exciting process with $n=1$, which means that we have a critical dynamical regime, as we have seen, but with different background 
rates, one much smaller than 1 and the other much greater than 1. This fact will be reflected in the percolation diagram. We will not approach these cases from a theoretical point of view
but from a graphical one. With the second configuration, as we have seen in Figure \ref{f: Hawkes rate burst} if the condition $\mu\ll 1$ is satisfied, we will have a bursty structure in the 
time series. Due to the low background rate, the events are less likely to occur, but when they do, they tend to form avalanches of activity thanks to the self-excitation. This will be
reflected in the percolation diagram as a first phase transition at a critical point $\Delta^*_1$ when $\Delta$ is of the order of the average cluster size. Then, a second phase transition
will occur at a critical point $\Delta^*_2$ when $\Delta$ is greater than the greatest inter-event time. This phenomenon is illustrated in Figure \ref{f: Delta percolación}.

\begin{figure}[H]
    \centering
    \includegraphics[width=0.9\textwidth]{Raster plot deltas.png}
    \caption{Diagram for $\mu\ll 1$. Red lines represent the events, clusters are coloured. As we can see, we have two regimes, one when $\Delta$ is of the order of the average 
    cluster size and another when it is of the order of the inter-event time where the system percolates.}
    \label{f: Delta percolación}
\end{figure}

On the other hand, when $\mu\gg 1$ events occur more frequently, without making the bursty structure of Figure \ref{f: Hawkes rate burst}, but making a more regular structure as
illustrated in Figure \ref{f: Hawkes rate burst 2}. This will be reflected in the phase diagram as a single phase transition at a critical point $\Delta^*$ when $\Delta$ is of the order of
the average cluster size. This phenomenon is illustrated in Figure \ref{f: Delta percolación 2}. Note the absence of a time scale in both diagrams, they are examples for the explanation of 
the phase diagram.

\begin{figure}[H]
    \centering
    \includegraphics[width=0.9\textwidth]{Raster plot deltas 2.png}
    \caption{Diagram for $\mu\gg1$. Red lines represent the events, clusters are coloured. In this situation, events occur more regularly, resulting in a unique percolation transition, 
    due to $\Delta$ being approximately equal to the inter-event time.}
    \label{f: Delta percolación 2}
\end{figure}