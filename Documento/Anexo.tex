\chapter*{Anexo}
\begin{lstlisting}[caption = {Algoritmo.}, label = {lst:}]
def algorithm(rate, mu, n):
"""
Algorithm that computes interevent times and Hawkes intensity

#Output: rate x_k, x_k
"""                                    
# Paso 1
u1 = np.random.uniform()
if mu == 0:
    F1 = np.inf
else:
    F1 = -np.log(u1) / mu

# Paso 2
u2 = np.random.uniform()
if (rate - mu) == 0:
    G2 = 0
else:
    G2 = 1 + np.log(u2) / (rate - mu)
    

# Paso 3
if G2 <= 0:
    F2 = np.inf
else:
    F2 = -np.log(G2)

# Paso 4
xk = min(F1, F2)

# Paso 5
rate_tk = (rate - mu) * np.exp(-xk) + n + mu
return rate_tk, xk 
\end{lstlisting}

\begin{lstlisting}[caption = {Generar series.}, label = {lst:}]
def generate_series(K, n, mu):
"""
Generates temporal series for K Hawkes processes

##Inputs:
K: Number of events
n: Strength of the Hawkes process
mu: Background intensity 

##Output:
times: time series the events
rate: time series for the intensity
"""
times_between_events = [0]
rate = [mu]
for _ in range(K):
    rate_tk, xk = algorithm(rate[-1], mu, n)
    rate.append(rate_tk)
    times_between_events.append(xk)
times = np.cumsum(times_between_events)
return times, rate
\end{lstlisting}

\begin{lstlisting}[caption = {Identify clusters.}, label = {lst:}]
def identify_clusters(times, delta):
"""
Identifies clusters in a temporal series given a resolution parameter delta

## Inputs:
times: temporal series
delta: resolution parameter

## Output:
clusters: list of clusters
"""
clusters = []
current_cluster = []
for i in range(len(times) - 1):
    if times[i + 1] - times[i] <= delta:
        if not current_cluster:
            current_cluster.append(times[i])
        current_cluster.append(times[i + 1])
    else:
        if current_cluster:
            clusters.append(current_cluster)
            current_cluster = []
return clusters
\end{lstlisting}

\begin{lstlisting}[caption = {Series perc.}, label = {lst:}]
def generate_series_perc(K, n, mu):
"""
Generates temporal series for K Hawkes processes

##Inputs:
K: Number of events
n: Strength of the Hawkes process
mu: Background intensity 

##Output:
times_between_events: time series the interevent times
times: time series the events
rate: time series for the intensity
"""
times_between_events = [0]
rate = [mu]
for _ in range(K):
    rate_tk, xk = algorithm(rate[-1], mu, n)
    rate.append(rate_tk)
    times_between_events.append(xk)
times = np.cumsum(times_between_events)
return times_between_events, times, rate
\end{lstlisting}

\begin{lstlisting}[caption = {Percolations.}, label = {lst:}]
def calculate_percolation_strength(times_between_events, deltas):
percolation_strengths = []

for delta in deltas:
    cluster_sizes = []
    # Initialize the size of the current cluster
    current_cluster_size = 1 # The first event is always a cluster

    for i in range(len(times_between_events)):
        if times_between_events[i] <= delta:
            current_cluster_size += 1
        else:
            if current_cluster_size > 1: # Only consider clusters with more than one event
                cluster_sizes.append(current_cluster_size)
            # Reset the size of the current cluster
            current_cluster_size = 1 # The next event is always a cluster

    # Add the size of the last cluster
    if current_cluster_size > 1: # Only consider clusters with more than one event
        cluster_sizes.append(current_cluster_size)

    max_cluster_size = max(cluster_sizes) 

    percolation_strengths.append(max_cluster_size / len(times_between_events))
return percolation_strengths
\end{lstlisting}
\begin{lstlisting}[caption = {Modelo.}, label = {lst:}]
def model(n_max, mu_E, mu_I, tau, n_EE, n_IE, n_EI, n_II, dt):
"""
Solve the equations of the mena field model for a given number of iterations n_max

Inputs:
n_max: number of iterations
mu_E: Poisson rate of excitatory neurons
mu_I: Poisson rate of inhibitory neurons
tau: characteristic time of the system
n_EE: influence of excitatory neurons on excitatory neurons
n_IE: influence of excitatory neurons on inhibitory neurons
n_EI: influence of inhibitory neurons on excitatory neurons
n_II: influence of inhibitory neurons on inhibitory neurons
dt: time step

Outputs:
t_events_E: times of events of excitatory neurons
t_events_I: times of events of inhibitory neurons
rates_E: rates of excitatory neurons
rates_I: rates of inhibitory neurons
"""
n_E = n_I = n = 0
t_events_E = [0]
t_events_I = [0]
rates_E = [mu_E]
rates_I = [mu_I]
time = [0]
while n <= n_max:
    # Excitation neurons
    l_Enew = rates_E[-1]  + dt * (mu_E- rates_E[-1])/tau
    if np.random.uniform() < rates_E[-1]*dt:
        l_Enew += n_EE
        t_events_E.append(t_events_E[-1]+dt*np.random.uniform())
        n_E += 1
    if np.random.uniform() < rates_I[-1]*dt:
        l_Enew -= n_IE
        t_events_E.append(t_events_E[-1]+dt*np.random.uniform())
        n_E += 1

    # Inhibition neurons
    l_Inew = rates_I[-1] + dt * (mu_I- rates_I[-1])/tau
    if np.random.uniform() < rates_E[-1]*dt:
        l_Inew += n_EI
        t_events_I.append(t_events_I[-1]+dt*np.random.uniform())
        n_I += 1
    if np.random.uniform() < rates_I[-1]*dt:
        l_Inew -= n_II
        t_events_I.append(t_events_I[-1]+dt*np.random.uniform())
        n_I += 1
    rates_E.append(l_Enew)
    rates_I.append(l_Inew)
    time.append(time[-1]+dt)

    n = n_E + n_I
return time, t_events_E, t_events_I, rates_E, rates_I
\end{lstlisting}